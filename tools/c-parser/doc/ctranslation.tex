%
% Copyright 2020, Data61, CSIRO (ABN 41 687 119 230)
%
% SPDX-License-Identifier: BSD-2-Clause
%

\documentclass{article}

\usepackage{alltt}
\usepackage{underscore}
\usepackage{url}
\usepackage{stmaryrd}

\newcommand{\strictc}{\textsf{StrictC}}
\title{Translation of the \strictc{} Dialect}
\author{Michael Norrish}
%
% Copyright 2020, Data61, CSIRO (ABN 41 687 119 230)
%
% SPDX-License-Identifier: CC-BY-SA-4.0
%

\renewcommand\today{\number\day~\ifcase\month\or
  January\or February\or March\or April\or May\or June\or
  July\or August\or September\or October\or November\or December\fi
  \space \number\year}


\newcommand{\MLsuffix}{.ML}
\newcommand{\srcfile}[1]{\texttt{#1}}
\newcommand{\MLfile}[1]{\srcfile{#1\MLsuffix}}
\newcommand{\eg}{\textit{e.g.}}
\newcommand{\ie}{\textit{i.e.}}
\newcommand{\etc}{\textit{etc.}}
\newcommand{\globexnvar}{\texttt{global\_exn\_var}}
\newcommand{\globheap}{\texttt{global\_heap}}
\newcommand{\naive}[1]{na\"\i{}ve#1}
\newcommand{\Sem}[1]{\ensuremath{\llbracket#1\rrbracket}}

\begin{document}
\maketitle
\tableofcontents

\section{Introduction}
\label{sec:introduction}

This report describes the translation program that imports C source into
a running Isabelle/HOL process, making a series of Isabelle
definitions in the process, as well as discharging a number of
(relatively minor) proof obligations that arise along the way.

The source code for the translator is found in the directory \texttt{c-parser}.
Source files, \eg, \MLfile{file}, are found in this directory unless otherwise noted.

The translation expects its input to be a well-formed C source file.
Such a source file must additionally satisfy a number of other
constraints, giving rise to a subset of C that is here called
``\strictc{}''.  Files in \strictc{} may also include special
annotations intended only for consumption by Isabelle (and the human
code-verifier).

In fact, there are two important \strictc{} programs: the translation
program, and the analysis program.  The analysis program is entirely
independent of Isabelle, and can be used to check that a source file
conforms to the \strictc{} subset.  It also implements a number of
analyses that can be performed on source code.  For example, it can
output the input file's call-graph, and can list the globals that are
read or modified in each function.  The additional source-files
supporting the analysis program are found in
\begin{verbatim}
   c-parser/standalone-parser
\end{verbatim}

The rest of this report describes both the functionality provided by
these programs (focussing on the translator), how this functionality
is implemented, and \emph{where} it is implemented.  The aim is to
give a picture of the systems' design in a way that should make future
modification of the code possible.

\subsection{\strictc{} Subset Summary}
\label{sec:subset-summary}

This is a brief list summarising the simplest restrictions imposed by
the \strictc{} subset.
\begin{itemize}
\item No \texttt{goto} statements.
\item No fall-through cases in \texttt{switch} statements.  Cases can
  be terminated with \texttt{continue} and \texttt{return}, as well as
  \texttt{break}.
\item Labels for \texttt{switch} statement cases must all appear at
  the syntactic level immediately below the block statement that must
  appear below the \texttt{switch} statement.
\item No unions.  (These are handled by a separate tool; see
  Cock~\cite{Cock08}.)
\item No \texttt{struct}, \texttt{enum} or \texttt{typedef}
  declarations anywhere except at the top, global, level.
\end{itemize}

\section{Abstract Syntax}
\label{sec:abstract-syntax}

The core data types in the translator represent the input program.
These abstract syntax values are the product of parsing the concrete
syntax (see Appendix~\ref{sec:grammar} for the grammar used), and is
the subsequent input to all further analyses and translation.  The
abstract syntax declarations are given in \MLfile{Absyn}.  For
example, the definition of the syntax type corresponding to C
statements is given in Figure~\ref{fig:stmt-absyn}.  There are also
definitions for C types, expressions, and declarations in
\MLfile{Absyn}.

The type \texttt{statement} is actually a \texttt{statement_node}
wrapped inside a ``region'' (see \MLfile{Region} and
Section~\ref{sec:regions} below), which provides information about
where the original concrete syntax originated.  This is used for
providing error messages.
\begin{figure}[htbp]
\begin{verbatim}
   datatype statement_node =
         Assign of expr * expr
       | AssignFnCall of expr option * expr * expr list
       | EmbFnCall of expr * expr * expr list
       | Block of block_item list
       | While of expr * string wrap option * statement
       | Trap of trappable * statement
       | Return of expr option
       | ReturnFnCall of expr * expr list
       | Break
       | Continue
       | IfStmt of expr * statement * statement
       | Switch of expr * (expr option list * block_item list) list
       | EmptyStmt
       | Auxupd of string
       | Spec of ((string * string) * statement list * string)
       | AsmStmt of {volatilep : bool, asmblock : asmblock}
   and statement = Stmt of statement_node Region.Wrap.t
   and block_item =
       BI_Stmt of statement
     | BI_Decl of declaration wrap
\end{verbatim}
\caption{The Abstract Syntax Data Type for C Statements}
\label{fig:stmt-absyn}
\end{figure}

All strings in the statement declaration correspond to Isabelle terms
(\eg, the loop invariant in the \texttt{While} case).
These will be parsed as such later in the translation
process, but are just uninterpreted strings when the C parser
finishes.
%% FIXME: this had been here for a long time but is not currently true.
%%        See VER-503 and l4v PR #329
%The statement type illuminates \strictc's most radical departure from
%ISO~C: the movement of assignment and function calls from the category
%of expression to statement.
%\footnote{As is not documented but should be, function calls may seem
%as if they are permitted inside expressions, but are not really.}
%This change syntactically enforces the requirement that expressions must be
%side-effect free.  Both forms must occur at the top-level.
Function calls can return their results into l-values, have the return
value ignored, or have the return value itself \texttt{return}-ed.
The first option corresponds to having the \texttt{expr~option} argument
be \texttt{SOME~e} in the \texttt{AssignFnCall} constructor, the second
would have that parameter be \texttt{NONE}, and the last is handled by
the \texttt{ReturnFnCall} constructor.

Syntactically, these options correspond to writing
\begin{verbatim}
   var = f(x,y);
\end{verbatim}
or
\begin{verbatim}
   f(x,y);
\end{verbatim}
or
\begin{verbatim}
   return f(x,y);
\end{verbatim}

\paragraph{C99 Block Items}
Conforming to the C99 grammar, the input language allows declarations
at any point inside a block, not just in a sequence at the head of the
block.  In other words, the following
\begin{verbatim}
   {
     x = f(z);
     int y = x + 1;
     while (x < y) { ... }
   }
\end{verbatim}
is legal in C99.  This means that a block has
to take a list of \texttt{block_item} values as an argument, where a
\texttt{block_item} is either a statement or a declaration.

\paragraph{Syntactic Sugar for Loops}
The abstract syntax has just one form of loop, the \texttt{While}
constructor.  The optional string argument to \texttt{While} is used
to represent any user-supplied invariant.  Together with the
\texttt{Trap} constructor, this is used to implement all three forms
of C loop.  The translation follows the model from Norrish's
PhD~\cite[p60]{norrish98}.  It also supports \texttt{for}-loops with
declarations in the first position.  This latter, for example,
\begin{verbatim}
   for (int i = 3; i < 10; i++) ...
\end{verbatim}
is a feature of C99.

Breaking from C, the grammar has the third component of the
\texttt{for} loop form be a restricted form of statement, rather than
an expression.  The parser syntax only allows comma-separated
increments (\ie, \texttt{++}), decrements and assignments.

The Isabelle translation eventually compiles all loops to one
underlying the loop primitive in the VCG environment called
\texttt{While}.  This form does not handle exceptional control-flow
forms like \texttt{break} and \texttt{continue}.  These are instead
handled by the \texttt{Trap} constructor, mapping to the VCG
language's \texttt{TRY}-\texttt{CATCH} form.


\subsection{Regions}
\label{sec:regions}

The \texttt{Region} module implements a method for annotating
arbitrary data types with location information.  This module has been
taken from the \texttt{MLton} compiler project (which has a BSD-style
open source licence).  It is used a great deal in the system, and its
use could probably be extended still further.  Region information is
used to produce good error messages.

The basic type is that of the \texttt{region} which is essentially a
pair of ``source positions'' (which are in turn implemented in
\MLfile{SourcePos}).  One useful source position is
\texttt{SourcePos.bogus}, corresponding to ``nowhere'' (perhaps
because some syntax has been conjured out of nowhere and doesn't
really exist in a file).

Regions are then used to implement the concept of a ``wrap'' (SML type
\texttt{Region.Wrap.t}), a polymorphic data type.  The file
\MLfile{Absyn} declares the following abbreviation:
\begin{alltt}
   type 'a wrap = 'a Region.Wrap.t
\end{alltt}
An \texttt{'a~wrap} (read ``alpha wrap'') is an \texttt{'a} value
coupled with a region.  The important functions for manipulating wraps
are
\begin{alltt}
   val wrap    : 'a * SourcePos.t * SourcePos.t -> 'a wrap
   val bogwrap : 'a -> 'a wrap
   val left    : 'a wrap -> SourcePos.t
   val right   : 'a wrap -> SourcePos.t
   val node    : 'a wrap -> 'a
   val apnode  : ('a -> 'b) -> 'a wrap -> 'b wrap
\end{alltt}
For example, the grammar code in \srcfile{StrictC.grm} manipulates a number
of values of type \texttt{string~wrap}.  If an error relating to this
value arises, both the string and its position can be reported to the
user.

Things become more complicated when the type to be wrapped is
recursive.  The standard idiom in the project is illustrated with the
definition of the \texttt{statement} data type (shown in
Figure~\ref{fig:stmt-absyn}).  The constructors for values of the type
are actually given in an auxiliary type (\texttt{statement_node}),
but the recursive constructors take arguments of type
\texttt{statement}.  The type \texttt{statement} is then a type that
is mutually recursive with \texttt{statement_node}, and which has
just one constructor, \texttt{Stmt}.\footnote{It would be nice if one
  could just declare \texttt{statement} to be an abbreviation of
  \texttt{statement_node~wrap}, but SML doesn't permit this.  It must
  be a \texttt{datatype} itself, and thus must have at least one
  constructor.}

Because a \texttt{statement} is not a wrap, the project cannot
directly call functions like \texttt{node} on \texttt{statement}
values.  Instead, helper functions are declared:
\begin{alltt}
   val sleft  : statement -> SourcePos.t
   val sright : statement -> SourcePos.t
   val snode  : statement -> statement_node
   val swrap  : SourcePos.t * SourcePos.t * statement_node ->
                statement
\end{alltt}
When code wishes to pattern-match against the multiple possible forms
a statement \texttt{s} may have, the idiom is to write
\begin{alltt}
   case snode s of
     EmptyStmt => ...
   | While(g,inv,body) => ...
   | IfStmt(g,ts,es) => ...
   | ...
\end{alltt}
The strength of this idiom is that one \emph{always} manipulates
values of type \texttt{statement}.  In particular, if the case
analysis above is to make recursive calls of its analysis on
sub-statements such as \texttt{body}, \texttt{ts} and \texttt{es},
these values are of the correct type for this to be done immediately.

The \texttt{expr} (expression) type (home to constructors such as
\texttt{Deref}, \texttt{Var} and \texttt{TypeCast}) is set up in the
same style, giving rise to functions \texttt{eleft}, \texttt{eright},
\texttt{enode}, \texttt{ewrap} and \texttt{ebogwrap}.

\paragraph{The type of C types}
The type of C types is \texttt{'a ctype}, with constructors such as
\texttt{Void} and \texttt{Ptr}.  Though recursive, it doesn't use the
wrap idiom.  On the other hand, this type is polymorphic.  The
\texttt{'a} parameter is instantiated with an SML type that
corresponds to the forms that give the size of arrays when they are
declared.  This type parameter is instantiated with \texttt{expr} when
the input file is first parsed.  In this way, a declaration like
\begin{verbatim}
   unsigned char array[EnumConst1 * sizeof(int*)];
\end{verbatim}
can be handled.  Thus, the \texttt{VarDecl} constructor of the
\texttt{declaration} type
\begin{verbatim}
   val VarDecl :
         expr ctype * string wrap * bool * initializer option ->
         declaration
\end{verbatim}
takes an \texttt{expr~ctype} as its first parameter.  Within
subsequent phases of the analysis and translation, it is much more
convenient to work with values of type \texttt{int~ctype}, where the
(constant) expression has been evaluated.  For example, the
\texttt{get_rettype} function from \MLfile{program_analysis},
takes a \texttt{csenv} value (see Section~\ref{sec:symbol-table}
below) and the name of a function, and returns the return type of the
function.  The value returned is an \texttt{int~ctype}.  The
conversion of an \texttt{expr~ctype} into an \texttt{int~ctype} is done
by the function \texttt{Absyn.constify_abtype}.

\section{The Symbol Table}
\label{sec:symbol-table}

All of the work done in analysis and translation of \strictc{} revolves
around the information stored in two important data structures
implemented in the module \MLfile{program-analysis}.  The first of
these is the \texttt{var_info} type.  This stores information about
individual variables.  The second type, \texttt{csenv} (``C state
environment''), stores information about the program as a whole,
including its collection of variables, but also recording details such
as where variables are read and modified, and the program's call-graph
structure.

The \texttt{var_info} type stores information about declared
identifiers living in the name-space that encompasses normal objects,
functions and enumeration constants.  In addition to the type of the
variable (\eg, \texttt{int}, \texttt{char~*} \etc), the
\texttt{var_info} also includes information about where the variable
was declared in terms of program locations, and also in terms of scope
(it might be global, or declared local to a particular function).

The \texttt{csenv} type accumulates its information about the program
by performing traversals of the abstract syntax tree.  \textbf{The
  \strictc{} translator makes no effort to be a one-pass compiler}, but
the number of traversals is no greater than three, and will probably
be reduced in future versions of the implementation.  These traversals
are performed after the parser has constructed all of the tree.  There
are also places in this analysis where \textbf{the translator assumes
  that it has seen the whole program}.  In particular, the translator
cannot be used to translate \emph{translation units} that are to be
separately compiled.  It must be presented with a concatenation of the
complete sources.

The bulk of the API for manipulating values of type \texttt{csenv} is
concerned with pulling information out of the symbol table.  For
example, it is possible to calculate the type of a C expression with
the function
\begin{verbatim}
   val cse_typing : csenv -> expr -> int ctype
\end{verbatim}

In addition, \MLfile{program-analysis} contains the one entry-point
for taking a sequence of external declarations (once parsed) and
creating a \texttt{csenv} value:
\begin{verbatim}
  val process_decls :
    Absyn.ext_decl list ->
    ((Absyn.ext_decl list * Absyn.statement list) * csenv)
\end{verbatim}
The return type includes a modified version of the syntax that was
provided as input, a list of the initialising assignments for the
global variables, and the \texttt{csenv} value.

\subsection{Functional Record Updates in SML}
\label{sec:fru}

The code in \MLfile{program-analysis} uses a powerful, but cryptic
SML idiom that makes it easy to define SML records along with
functions for updating their fields.  Done \naive{ly}, writing code to
do this represents a quadratic amount of work for the programmer.  Put
another way, the \naive{} approach requires $O(n)$ much typing
whenever a field is added to or removed from a record definition of
$n$ fields.

The cryptic technique is fully described at
\begin{alltt}
   \url{http://mlton.org/FunctionalRecordUpdate}
\end{alltt}
and allows the addition or deletion of a field to be done with $O(1)$
much typing.  Supporting code is in \MLfile{FunctionalRecordUpdate}.

The cryptic code is isolated within \MLfile{program-analysis},
where it is used to define update functions that are subsequently used
exclusively.  In general, when one of the two types has a field
\texttt{fld} of type $\tau$, then there will typically be a function
$\{\texttt{cse}|\texttt{vi}\}$\texttt{_fupd_fld} defined, with type
\[
  (\tau \rightarrow \tau) \rightarrow \textsf{rcd} \rightarrow
  \textsf{rcd}
\]
where \textsf{rcd} is either \texttt{var_info} or \texttt{csenv}.
Such functions can be used to update the fields of a record: the user
provides a function that is given the old value of the field, and
which returns the new value.



\section{Creation of the Hoare Environment State}
\label{sec:hoare-state}

The major oddity about Norbert Schirmer's Hoare environment, into which
we are translating our programs, is that all local variables, including
function parameters, have to be part of the ``global state''.  This
state must be declared before any functions can be translated,
because a function becomes an Isabelle definition (conceptually at
least) that operates over that state space.

Slightly simplifying, the state space is an Isabelle type that is a
record with fields for every local and global variable.  Each field
has a type (Isabelle/HOL is a typed logic after all), which means that
all local variables of the same name in the same \strictc{} translation
unit must have the same type.  This is rather an arbitrary
requirement, but easy both to enforce and to comply with.  Thankfully,
signed and unsigned variants of the same underlying type (such as
\texttt{signed~short} and \texttt{unsigned~short}) are given the same
Isabelle/HOL type, so there is a little leeway.  Nonetheless, if
\texttt{i} is of type \texttt{int} in one function, it can not be of
type \texttt{char} in another.\footnote{If this is attempted, the
  system will ``munge'' one of the variables so that it has a
  different name when translated into Isabelle.  The ``munged'' name
  is stored in the variable's \texttt{var_info} record.}

The arrangement of global and local variables is actually slightly
complicated.  In essence, the state-space is set up to look like:
\begin{verbatim}
   statespace = record
      globals :: global_var_type
      local_var1 :: lvar1_type
      local_var2 :: lvar2_type
      ...
      local_varn :: lvarn_type
\end{verbatim}
where the field \texttt{globals} is of a custom record type
\texttt{global_var_type} that in turn contains all of the global
variables.  In addition to the user program's own globals, the
translation process adds two extra global variables of its own.  These
variables are used for handling ``exceptional exits'' (such as those
caused by the \texttt{break}, \texttt{continue} and \texttt{return}
statements), and for modelling the global heap.  The exact names of
these variables is not important, here we will refer to them as
\globexnvar, and \globheap.

These two special variables are part of \texttt{global_var_type} and
so must have Isabelle types themselves.  The type of \globexnvar{} is
the enumerated type \texttt{c_exntype}, defined in the Isabelle
theory \texttt{CProof} to have three possible values, \texttt{Break},
\texttt{Return} and \texttt{Continue}.  The type of the global heap
\globheap{} field is a product of underlying heap contents (a map of
type $\texttt{addr}\rightarrow\texttt{word8}$) and a special-purpose
data type to store type information about the heap memory (see Harvey
Tuch's PhD thesis~\cite{Tuch:phd} for more on this).

\subsection{Representing Values in Memory}
There is one important requirement that must be met by all object
types that occur in C programs: they must be representable in memory.
Alternatively, it must be possible to encode values of the types in a
program as sequences of bytes, and to then decode those same bytes
back into the original values.  One approach to modelling this
fundamental requirement might be to have Isabelle functions that
manipulated only lists of bytes.  Working at the level of this
untyped, and very concrete, view of the program state would be an
extremely poor state of affairs (the C programmer would have a more
abstract view of the program than the verifier).

Our approach is to use Isabelle type-classes to encode the fact that
an Isabelle type can be represented in a consistent amount of
``C~memory''.  When an Isabelle type \texttt{'a} is in the class
\texttt{mem_type}\footnote{See \texttt{umm_heap/CTypes.thy}.}, it
supports functions
\begin{verbatim}
   to_bytes   :: "'a::mem_type => byte list"
   from_bytes :: "byte list => 'a::mem_type"
\end{verbatim}
as well as a number of other supporting functions that record details
like the fields that occur in compound \texttt{struct} types, and the
(constant) length of the byte-lists that encode the values in the
type.

All of the atomic types manipulated by our programs are fixed-width
words (\eg, 32 bit words for integers).  It is easy to demonstrate
that these types are indeed in the \texttt{mem_type} class.  In
particular, we do \emph{not} pretend that programs manipulate infinite
precision integers, and then worry about whether or not these integers
can be pushed into and pulled out of memory.  All of the arithmetic
performed by the programs we verify is done at fixed widths,
respecting the underlying machine's operations.
Additionally, using the techniques from Section~\ref{sec:undefined-behaviour},
we trap the undefined behaviour caused by overflow on signed values.

\subsection{Pointers}
\label{sec:pointers}

Pointers are always represented as words of a particular size
(regardless of type being pointed to).  This is not required by the C
standard, which only requires pointers to \texttt{void} be capable of
storing all other non-function pointer values, and that all function
pointer values be inter-convertible.  Again, our decision to
specialise on particular, and reasonable, target architectures makes
life simpler.

Pointers retain type information by using ``phantom'' type variables.
The Isabelle declaration is
\begin{verbatim}
   datatype 'a ptr = Ptr addr
\end{verbatim}
Then, if the C program under analysis calls for a variable of type
\texttt{char~*}, the Isabelle environment will include a variable of
Isabelle type \texttt{byte~ptr}.  In this way our pointers are typed,
even if their underlying encoding makes it trivial to view a pointer
to one type as a pointer to another type.

Because the underlying representation is always the same, all pointer
types are proved to be in the class \texttt{mem_type} once and for
all (in theory \texttt{CTypes}).  Pointers to \texttt{void} are
represented as values in the Isabelle type \texttt{unit~Ptr}, where
\texttt{unit} is the standard singleton type.  The \texttt{unit} type
is not shown to be in the class \texttt{mem_type}.

\subsection{Arrays}
\label{sec:arrays}

C arrays are lists of values of fixed length.  A faithful
representation of this type requires a novel Isabelle type.  We build
on Anthony Fox's implementation of John Harrison's ``finite Cartesian
products'' idea~\cite{Harrison:TPHOLs2005:Euclidean-Space}.  Syntactic
trickery within Isabelle allows us to write types like
\begin{verbatim}
   nat[10]
\end{verbatim}
which is an array of 10 natural numbers.  There are operators for
updating and indexing into arrays.  Note that the type \texttt{10}
that appears above is an Isabelle \emph{type}, not a term.

If type $\tau$ is a \texttt{mem_type}, then an array of $\tau$ values
will also be a \texttt{mem_type}, as long as the size of the array is
not so large that the array would not fit into memory.  This condition
is discharged as the \strictc{} program is translated.

For technical reasons due to the implementation of type classes in
Isabelle, we need to fix separate limits, ahead of time, on the
number of elements in an array and the size of each element.
Currently, for 32-bit ARM, our model fixes a maximum of:

\begin{itemize}
\item $2^{13}$ (8192) elements in each array; and
\item $2^{19}$ bytes ($512$~KiB) in each array element
\end{itemize}

For x86-64, the limits are:

\begin{itemize}
\item $2^{20}$ ($1048576$) elements in each array; and
\item $2^{26}$ bytes ($64$~MiB) in each array element
\end{itemize}

One would prefer to be able to multiply the size of the element type by
the number of the elements, but the type system does not permit this
(for good technical reasons).

\subsection{C \texttt{struct} Types}
\label{sec:structs}

From the point of view of the translation into Isabelle,
\texttt{struct} types do not pose any great conceptual difficulties.
A \texttt{struct} type is clearly very similar to an Isabelle record,
which in turn is conceptually the same as a tuple.  The first
complication that arises is that Isabelle tuples can not be recursive,
whereas C \texttt{struct} types are often recursive (as when
implementing linked structures in the heap).

This required the implementation of an alternative record definition
package, allowing (possibly multiple) recursive types.  This then
allows Isabelle types to be defined that correspond to C declarations
such as those shown in Figure~\ref{fig:struct-examples}.

\begin{figure}
\begin{minipage}[t]{0.4\textwidth}
\begin{verbatim}
   struct listnode {
      int node_data;
      struct listnode *next;
   };
\end{verbatim}
\end{minipage}
\qquad\quad
\begin{minipage}[t]{0.4\textwidth}
\begin{verbatim}
   struct node2;
   struct node1 {
      struct node2 *data;
      struct node1 *next, *prev;
      int someflag;
   };
   struct node2 {
      struct node1 *owner;
      char stringdata[100];
   };
\end{verbatim}
\end{minipage}
\caption{Examples of Recursive \texttt{struct} Declarations Accepted
  in \strictc}
\label{fig:struct-examples}
\end{figure}

Confirming that a \texttt{struct} type really is representable in
memory, requires the definition of functions for converting Isabelle
records into lists of bytes and \emph{vice versa}.  The size of the
converted value must also be checked to be no bigger than the size of
memory.  Both of these actions require knowledge of how the fields of
the \texttt{struct} are laid out in memory, which is in turn a
function of the padding that can be inserted between the fields.  Such
calculations are architecture dependent.


\section{Translating Expressions}
\label{sec:expressions}

Expression translation is implemented in
\MLfile{expression\_translation}.

\paragraph{Fundamental Concepts}
\strictc{} expressions are essentially a subset of C expressions, and
are fairly easy to translate to corresponding Isabelle expressions
that manipulate Isabelle-encoded values.  There are two fundamental
concepts to grasp of expression translation.  First, when being
evaluated (``read to determine a value'') a C expression of type
$\tau$ becomes an Isabelle expression of type $\texttt{statespace}
\rightarrow \Sem{\tau}$, where $\Sem{\tau}$ is the translated,
Isabelle version of C type $\tau$.

This must be done to make sense of expressions that read memory: an
expression such as \texttt{s.arrayfld[3]} only has a specific value in
the context of a specific state of memory.  This use of a ``lifted''
function space to represent the expression is a standard technique in
denotational semantics.  In the example given, the value of the
expression is a function that looks at the statespace to determine
what data is stored at variable \texttt{s}.  As the statespace
evolves, the value returned from this function changes, but the
function's value is the same.

Expressions do not just determine values however, they can also denote
an \emph{l-value}, something denoting a ``place in memory'' that is to
be updated.  This is done when an address is taken, or when an
assignment is to be performed.  In a simple language, l-values might
only be variable names, but C allows for complicated expressions on
both sides of an assignment.  The example above
(\texttt{s.arrayfld[3]}) might just as well be assigned to as read.
So, the l-value of an expression $e$ that has type $\tau$ will be an
Isabelle value of type $\texttt{statespace} \rightarrow
\tau\rightarrow\texttt{statespace}$, a function that takes a new
value and a statespace to change, and returns the updated statespace.

Not all expressions are l-values (the expression \texttt{3} is not,
for example), so the translation of any one expression can return one
or two different values, an ``r-value'' as well as an optional
l-value.

In addition, because of the way the translation does not put local
variables into memory, the translation provides another separate
optional value, that of the expression's address.  If everything did
live in memory, all l-values would have addresses, and one could
dispense with the separate calculation of l-values.  Thus the first
three lines of Figure~\ref{fig:exprinfo}.

\begin{figure}
\begin{verbatim}
   datatype expr_info =
     EI of {lval   : (term -> term -> term) option,
            addr   : (term -> term) option,
            rval   : term -> term,
            cty    : int ctype,
            ibool  : bool,
            lguard : (term -> term * term) list,
            guard  : (term -> term * term) list,
            left   : SourcePos.t,
            right  : SourcePos.t }
\end{verbatim}
\caption{The \texttt{expr_info} type, into which C expressions are
  translated internally.}
\label{fig:exprinfo}
\end{figure}

The SML types given to the \texttt{lval}, \texttt{addr} and
\texttt{rval} fields in the declaration of \texttt{expr_info} are
themselves function spaces at the SML level.  These function spaces
manipulate values of SML type \texttt{term}.  The way in which typing
at the C level is reflected at the Isabelle level is mainly hidden at
the SML level, where the programmer just manipulates terms (the
Isabelle types are internal to those values).

However, the function spaces \emph{can} be made visible at the SML
level.  This is done mainly to reduce the number of $\beta$-redexes
that would otherwise be created in the resulting term.  For example,
translating the C expression \texttt{x~+~3} will first create r-values
\begin{eqnarray*}
\Sem{\texttt{x}} &=& (\lambda\sigma.\;\textsf{x}(\sigma))\\
\Sem{\texttt{3}} &=& (\lambda\sigma. \;3)
\end{eqnarray*}
where the application of the \textsf{x} function to a statespace
$\sigma$ pulls out the value of variable \texttt{x} in $\sigma$.

When translating an expression $e_1 + e_2$, one naturally creates the
value
\[
\lambda\sigma.\;\Sem{e_1}(\sigma) + \Sem{e_2}(\sigma)
\]
If this was done within Isabelle \texttt{term} values, the result
would be a \texttt{term} full of expressions of the form $(\lambda
v.\;M)N$.  By lifting the Isabelle $\lambda$ to the SML level, only
one abstraction need to be created, at the very top-level in the
translation.  Thus, the translation of the addition becomes the SML
expression
\begin{verbatim}
   (fn s => mk_plus (rval_of e1 s) (rval_of e2 s))
\end{verbatim}
where \texttt{rval_of} returns the \texttt{rval} field of an
\texttt{expr_info} and the $\beta$-reduction happens within SML.

The fourth line of the record in \texttt{expr_info} values stores the
type of the expression, something that informs the translation of many
different C expressions.  For example, additions are not as simple as
just presented because of the possibility that one of the arguments
might be a pointer.  The \texttt{left} and \texttt{right} fields of
the \texttt{expr_info} record store the source-code position of the
original expression.  The two guard fields are explained below in
Section~\ref{sec:undefined-behaviour}.

The \texttt{ibool} field of the \texttt{expr_info} records whether or
not the term being generated in the r-value is of Isabelle's boolean
type.  This is done so that translation can avoid some conversions
between Isabelle words and booleans.  For example, if the expression
being translated is \verb|x < 6 && y > 10|, the resulting Isabelle
term will include a use of the two comparison operators on words.
Strictly, one should then turn the boolean results of these operators
into either a one or zero.  But, as these results are then combined
with boolean conjunction, the values will immediately be converted
back into booleans.

Of course, in an expression such as \verb|x && x->fld > 6|, the first
operand to the conjunction does not have Isabelle boolean type, and
will be converted to a boolean value by comparing it with the null
pointer.  (In fact, this particular example causes a more complicated
translation effect to occur; see Section~\ref{sec:undefined-behaviour}
below on undefined behaviour.)  Dually, if the code puts a boolean
value into memory, the translation has to make sure that the Isabelle
term has the appropriate word type once more.  There are two functions
used here:
\begin{verbatim}
   mk_isabool : expr_info -> expr_info
   strip_kb   : expr_info -> expr_info
\end{verbatim}
The function \texttt{mk_isabool} produces an \texttt{expr_info} value
that does have Isabelle boolean type (it will be the identity function
on an r-value that is already known to be boolean).  The function
\texttt{strip_kb} reverses this, turning an Isabelle boolean into an
Isabelle word if necessary.

\subsection{Undefined Behaviour}
\label{sec:undefined-behaviour}

There are three classes of under-specification in the C standard.
Those classed as \emph{implementation defined} are behaviours or
values that are supposed to be fixed and documented by particular
implementations.  For example, the number of bits in an \texttt{int}
(a number that must be at least $16$), is a fixed value that implementations
are allowed to choose themselves.  Because \strictc{} targets a
particular architecture, most implementation-defined aspects of C can
be ``baked into'' the translation.  (The design attempts to have the
translation sources be easy to modify to account for different
architectures; see for example the \texttt{ImplementationNumbers}
structure in each of the \MLfile{TargetNumbers} files.)

The second class of under-specification comprises \emph{unspecified}
behaviours.  The most important of these is the order of evaluation of
arguments to operators and function calls.
%% FIXME: not true currently. See previous FIXME
%Our translation completely
%bypasses this problem by removing side effects from expressions,
%meaning that expressions can be evaluated in any order because they
%will all derive values based on the same underlying state.
Therefore, one should avoid writing code that depends on unspecified
evaluation order. Otherwise, the compiled executable might have
different semantics compared to the Simpl specification that we
generate. For example:
\begin{verbatim}
   int x = f() + g();
\end{verbatim}
where \texttt{f} and \texttt{g} both have side effects, is currently
allowed but should be avoided. In the future, we plan to forbid or
restrict programs that may have unspecified evaluation order.
The standalone analysis tool can check for this problem using the
\texttt{--embedded\_fncalls} option.

The third class of under-specification is \emph{undefined behaviour}.
In essence, undefined behaviours are runtime errors (such as
dereferencing a null pointer).  They are undefined because the
standard does not require the implementation to catch them, or to
realise that they have occurred.  Rather, if an undefined behaviour
occurs, the user can no longer rely on their program to do anything
sensible.  In effect, implementations are given licence to blunder on
however they like when an undefined behaviour occurs.

Thus, it is critical that the C code we verify never exhibits any
undefined behaviours.  We do this with a feature of the Hoare
environment called the \emph{guard}.  A guard is a boolean condition
$g$ attached to a statement $s$, with the combination written $g\to
s$.  If the guard $g$ is true when the combined statement is to be
executed, then $s$ is allowed to execute.  If it is false, the
underlying semantics defines the result to be a ``fault''.  When a
program faults, nothing more can happen, so it has effectively
aborted.  Verifying a program with guarded statements thus requires
proofs that the guards are always true.

Most guards arise in expressions, and the
translation process accumulates these in one top-level guard at the
statement level.  For example, in the statement
\begin{verbatim}
   x = *p >> i;
\end{verbatim}
there will be four guards attached to the statement that will need to
be discharged: that \texttt{p} is not null, that \texttt{*p} is not
negative, that \texttt{i} is not negative, and that \texttt{i} is less
than 32.

One elegant feature of guards in the Hoare environment is that they
can be selectively disabled for the purposes of verification.  For
example, the C standard's requirement that right shift operations might not
be performed on negative numbers (that it results in undefined
behaviour) is too strict, given a particular C compiler and target
architecture.  In this situation, it is possible to prove
Hoare-triples where that particular guard is not used, so that the
semantics of $g\to s$ is simply that of $s$.

%\subsection{Embedded Function Calls}
%\label{sec:embedded-function-calls}

%\section{Translating Statements}
%\label{sec:statements}

\appendix
\section{Concrete Syntax: Parsing and Lexing}
\label{sec:grammar}

The grammar for \strictc{} is given in the file \srcfile{StrictC.grm}, which
is given as input to the standard tool \texttt{mlyacc}.  The format of
an \texttt{mlyacc} file looks quite similar to the format accepted by
\texttt{yacc}.  A typical grammar rule is
\begin{verbatim}
   init_declarator_list
   : init_declarator
                   ([init_declarator])
   | init_declarator YCOMMA init_declarator_list
                   (init_declarator :: init_declarator_list)
\end{verbatim}
where a non-terminal appears before a colon, and multiple possible
right-hand sides are separated by the pipe or vertical bar character.
The code for a production appears in parentheses after each
right-hand-side.  The code's convention is to have token names (\eg,
\texttt{YCOMMA} above) be upper-case.

Apart from the changes that turn assignment expressions into
statements, the grammar for \strictc{} attempts to be as close as
possible to the grammar of the C standard.  In general, the names for
non-terminals in \srcfile{StrictC.grm} are taken from the standard, so it
should be fairly clear how the standard's grammar has been mapped into
\srcfile{StrictC.grm}.

\subsection{Lexing and \texttt{typedef} Names}
\label{sec:lexing-typedefs}

The standard problem in lexing and parsing C is that the grammar is
ambiguous: the non-terminal \textit{typedef-name} is defined to simply
be the same as an \textit{identifier}.  When the parser encounters
\begin{alltt}
   x * f(y);
\end{alltt}
it can't know if this is meant to be a multiplication of variable
\texttt{x} by a function call expression, or whether it is the
(prototype) declaration of a function called \texttt{f} taking an
argument of type \texttt{y} and returning a value of type \texttt{x}.

To resolve this, the lexer must be able to classify identifier tokens as
normal identifiers or \textit{typedef-names}.  The \strictc{}
translator's approach to this problem is not typical, because of the
strange way in which \texttt{mlyacc} combines handling of side effects
with its error correction.  Normally, one would have some sort of
updatable symbol table that the parser would write to when it
encountered a \texttt{typedef} declaration.  The lexer would read from
the same table as it encountered identifiers, allowing appropriate
categorisation.

The approach taken in the \strictc{} translator is to have the lexer do
all of the work, without reference to the parser (see
\srcfile{StrictC.lex}).  This \emph{is} possible, but is also convoluted,
and involves many updatable variables that are internal to the lexer.
The basic idea is that when the lexer sees a \texttt{typedef} token,
it switches to the \texttt{TDEF} state.  When an identifier is seen in
this state, the identifier is added to the list of
\textit{typedef-names}, and lexing can continue.  The complications
arrive in declarations like
\begin{verbatim}
   typedef struct s { int fld1; char fld2; } s_t;
\end{verbatim}
where the identifiers \texttt{s}, \texttt{fld1} and \texttt{fld2} have
to be ignored, and \texttt{s_t} taken as the new \emph{typedef-name}.
This requires the lexer to handle the matching brace characters, and
partly motivates the requirement that \texttt{typedef} declarations
all occur at the top-level (and not be nested).

\subsection{GCC \texttt{__attribute__} Declarations}
\label{sec:gcc-attributes}

The parser handles, but mostly ignores, various \texttt{gcc}-specific
extensions, such as \texttt{__attribute__}.  These are tricky to parse
(something admitted by the relevant \texttt{gcc} documentation): users
are given almost unlimited liberty to put their \texttt{__attribute__}
strings anywhere within a declaration.

\smallskip\noindent
For example, these three
\begin{alltt}
   int f(int) __attribute__((__const__));
   __attribute__((__const__)) int f(int);
   int __attribute__((__const__)) f(int);
\end{alltt}
are all supposed to parse successfully (and have the same meaning).
Making this work is rather involved.  Most attributes are ignored, but
the standalone analysis tool does check that \texttt{const} and
\texttt{pure} attributes are reasonable, given what it knows of how
functions may modify and read the global state.

\bibliographystyle{plain}
\bibliography{ctranslation}

\end{document}
